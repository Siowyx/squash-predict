% Options for packages loaded elsewhere
\PassOptionsToPackage{unicode}{hyperref}
\PassOptionsToPackage{hyphens}{url}
%
\documentclass[
]{article}
\usepackage{amsmath,amssymb}
\usepackage{iftex}
\ifPDFTeX
  \usepackage[T1]{fontenc}
  \usepackage[utf8]{inputenc}
  \usepackage{textcomp} % provide euro and other symbols
\else % if luatex or xetex
  \usepackage{unicode-math} % this also loads fontspec
  \defaultfontfeatures{Scale=MatchLowercase}
  \defaultfontfeatures[\rmfamily]{Ligatures=TeX,Scale=1}
\fi
\usepackage{lmodern}
\ifPDFTeX\else
  % xetex/luatex font selection
\fi
% Use upquote if available, for straight quotes in verbatim environments
\IfFileExists{upquote.sty}{\usepackage{upquote}}{}
\IfFileExists{microtype.sty}{% use microtype if available
  \usepackage[]{microtype}
  \UseMicrotypeSet[protrusion]{basicmath} % disable protrusion for tt fonts
}{}
\makeatletter
\@ifundefined{KOMAClassName}{% if non-KOMA class
  \IfFileExists{parskip.sty}{%
    \usepackage{parskip}
  }{% else
    \setlength{\parindent}{0pt}
    \setlength{\parskip}{6pt plus 2pt minus 1pt}}
}{% if KOMA class
  \KOMAoptions{parskip=half}}
\makeatother
\usepackage{xcolor}
\usepackage[margin=1in]{geometry}
\usepackage{color}
\usepackage{fancyvrb}
\newcommand{\VerbBar}{|}
\newcommand{\VERB}{\Verb[commandchars=\\\{\}]}
\DefineVerbatimEnvironment{Highlighting}{Verbatim}{commandchars=\\\{\}}
% Add ',fontsize=\small' for more characters per line
\usepackage{framed}
\definecolor{shadecolor}{RGB}{248,248,248}
\newenvironment{Shaded}{\begin{snugshade}}{\end{snugshade}}
\newcommand{\AlertTok}[1]{\textcolor[rgb]{0.94,0.16,0.16}{#1}}
\newcommand{\AnnotationTok}[1]{\textcolor[rgb]{0.56,0.35,0.01}{\textbf{\textit{#1}}}}
\newcommand{\AttributeTok}[1]{\textcolor[rgb]{0.13,0.29,0.53}{#1}}
\newcommand{\BaseNTok}[1]{\textcolor[rgb]{0.00,0.00,0.81}{#1}}
\newcommand{\BuiltInTok}[1]{#1}
\newcommand{\CharTok}[1]{\textcolor[rgb]{0.31,0.60,0.02}{#1}}
\newcommand{\CommentTok}[1]{\textcolor[rgb]{0.56,0.35,0.01}{\textit{#1}}}
\newcommand{\CommentVarTok}[1]{\textcolor[rgb]{0.56,0.35,0.01}{\textbf{\textit{#1}}}}
\newcommand{\ConstantTok}[1]{\textcolor[rgb]{0.56,0.35,0.01}{#1}}
\newcommand{\ControlFlowTok}[1]{\textcolor[rgb]{0.13,0.29,0.53}{\textbf{#1}}}
\newcommand{\DataTypeTok}[1]{\textcolor[rgb]{0.13,0.29,0.53}{#1}}
\newcommand{\DecValTok}[1]{\textcolor[rgb]{0.00,0.00,0.81}{#1}}
\newcommand{\DocumentationTok}[1]{\textcolor[rgb]{0.56,0.35,0.01}{\textbf{\textit{#1}}}}
\newcommand{\ErrorTok}[1]{\textcolor[rgb]{0.64,0.00,0.00}{\textbf{#1}}}
\newcommand{\ExtensionTok}[1]{#1}
\newcommand{\FloatTok}[1]{\textcolor[rgb]{0.00,0.00,0.81}{#1}}
\newcommand{\FunctionTok}[1]{\textcolor[rgb]{0.13,0.29,0.53}{\textbf{#1}}}
\newcommand{\ImportTok}[1]{#1}
\newcommand{\InformationTok}[1]{\textcolor[rgb]{0.56,0.35,0.01}{\textbf{\textit{#1}}}}
\newcommand{\KeywordTok}[1]{\textcolor[rgb]{0.13,0.29,0.53}{\textbf{#1}}}
\newcommand{\NormalTok}[1]{#1}
\newcommand{\OperatorTok}[1]{\textcolor[rgb]{0.81,0.36,0.00}{\textbf{#1}}}
\newcommand{\OtherTok}[1]{\textcolor[rgb]{0.56,0.35,0.01}{#1}}
\newcommand{\PreprocessorTok}[1]{\textcolor[rgb]{0.56,0.35,0.01}{\textit{#1}}}
\newcommand{\RegionMarkerTok}[1]{#1}
\newcommand{\SpecialCharTok}[1]{\textcolor[rgb]{0.81,0.36,0.00}{\textbf{#1}}}
\newcommand{\SpecialStringTok}[1]{\textcolor[rgb]{0.31,0.60,0.02}{#1}}
\newcommand{\StringTok}[1]{\textcolor[rgb]{0.31,0.60,0.02}{#1}}
\newcommand{\VariableTok}[1]{\textcolor[rgb]{0.00,0.00,0.00}{#1}}
\newcommand{\VerbatimStringTok}[1]{\textcolor[rgb]{0.31,0.60,0.02}{#1}}
\newcommand{\WarningTok}[1]{\textcolor[rgb]{0.56,0.35,0.01}{\textbf{\textit{#1}}}}
\usepackage{graphicx}
\makeatletter
\def\maxwidth{\ifdim\Gin@nat@width>\linewidth\linewidth\else\Gin@nat@width\fi}
\def\maxheight{\ifdim\Gin@nat@height>\textheight\textheight\else\Gin@nat@height\fi}
\makeatother
% Scale images if necessary, so that they will not overflow the page
% margins by default, and it is still possible to overwrite the defaults
% using explicit options in \includegraphics[width, height, ...]{}
\setkeys{Gin}{width=\maxwidth,height=\maxheight,keepaspectratio}
% Set default figure placement to htbp
\makeatletter
\def\fps@figure{htbp}
\makeatother
\setlength{\emergencystretch}{3em} % prevent overfull lines
\providecommand{\tightlist}{%
  \setlength{\itemsep}{0pt}\setlength{\parskip}{0pt}}
\setcounter{secnumdepth}{-\maxdimen} % remove section numbering
\ifLuaTeX
  \usepackage{selnolig}  % disable illegal ligatures
\fi
\IfFileExists{bookmark.sty}{\usepackage{bookmark}}{\usepackage{hyperref}}
\IfFileExists{xurl.sty}{\usepackage{xurl}}{} % add URL line breaks if available
\urlstyle{same}
\hypersetup{
  pdftitle={S\&DS 425 Report - Squash Match Prediction},
  pdfauthor={Yee Xian Siow},
  hidelinks,
  pdfcreator={LaTeX via pandoc}}

\title{S\&DS 425 Report - Squash Match Prediction}
\author{Yee Xian Siow}
\date{17 Dec 2023}

\begin{document}
\maketitle

\hypertarget{abstract}{%
\subsection{Abstract}\label{abstract}}

In this project, the objective is to forecast the outcomes of squash
matches between two players by leveraging historical match results
obtained from the Professional Squash Association website up to 26 Nov
2023. The dataset includes information on previous seasons' match
results, serving as the basis for training logistic regression models.
The predictors for these models are derived from the current game
scores, aiming to predict both game and match win probabilities. The
predictions generated by the models closely align with the current PSA
world rankings, showcasing the efficacy of logistic regression in
capturing patterns and dynamics within squash match outcomes. To
facilitate user interaction, a shiny app was developed, enabling users
to easily predict match outcomes for various players.

\hypertarget{introduction}{%
\subsection{Introduction}\label{introduction}}

With the recent inclusion in the 2028 Olympic Games, Squash has garnered
increasing attention, and predicting match outcomes is crucial for
players, fans, and organizers alike. The motivation behind this project
is to attempt to understand the squash match dynamics and develop
predictive models for game and match outcomes. The challenge addressed
is the inherent uncertainty in forecasting squash matches, considering
the multifaceted nature of player skills and strategies.

The dataset used in this project is sourced from the Professional Squash
Association website, encompassing match results from previous seasons up
to 26 Nov 2023. This comprehensive dataset includes player names,
seasons, game scores, and match outcomes, forming the basis for training
logistic regression models. The predictive variables are derived from
the ongoing game scores, providing a dynamic and real-time approach to
forecasting.

The next section contains data extraction and cleaning, which describe
the process of getting data from the Professional Squash Association
website using the Selenium package. Then, the data exploration section
reveals that some players only appeared in a few matches per season. In
the data modeling section, we build several different logistic
regression models for game and match win probabilities and found that
the model predicting game win probabilities has the lowest AIC. The
shiny section simply describe how the shiny app works. Then, we discuss
the results of the model, including the top 20 coefficients for the
players and the corresponding standard errors, in the results section.
Finally, we discuss conclusions, limitations, and ideas for future work
in the conclusions section.

\hypertarget{data-extraction-and-cleaning}{%
\subsection{Data Extraction and
Cleaning}\label{data-extraction-and-cleaning}}

We extracted matches results from the Professional Squash Association
website. First, we get the links to all completed tournaments results
page from all seasons from the tournaments page
(\url{https://www.psaworldtour.com/tournaments/}). The challenge for
this step is that the html table that display the tournaments from the
past seasons are generated dynamically, meaning the html code pulled
from the scan() function will only have the links to tournaments in the
current season. To solve this problem, we used the Selenium package,
which allows us to interact with elements on the webpage. This way, we
can select the specific season and extract the links for the
corresponding tournaments.

Once we have the list of all links, we extracted the relevant tournament
information and matches results from each link. To do this, we created a
function to navigate the html table that contains all information. There
were a lot of inconsistencies which results in a html table that has a
different format from most that need to be checked. For example, some
tournaments were only for one gender, tournaments have different number
of players, some results were not recorded, some tournaments used a
round robin format, some matches were best of 3 format, etc. The
resulting data frame contains information on players names, gender,
players tournament seeding, players countries, round (e.g.~Final), match
results (both games won and points won), tournament names, date of
match, tournament location, tournament prize money tier, and season.
Each row in the data frame represent a match. There were a total of
73498 matches results across more than 30 seasons extracted.

Next, we do some data cleaning which includes date formatting, ensuring
naming consistency for the players, and investigating NA entries. We
simply remove the rows with NA scores because this just meant that the
match results were not recorded on the website. For some of the NA
entries in the player seed and country columns, we were able to cross
check with other rows and fill in the correct values from another row
with matching players and tournaments.

Additionally, there were some extra data processing steps before fitting
the logistic regression models. The data were filtered based on seasons.
Then, a player coefficient matrix is created, where each column
represent a player and the entries were set to 1 if the player is
player1, -1 if the player is player2, else 0.

For the first model, which predicts match win probability, the data
frame used includes the player coefficient matrix and a column
indicating if player1 wins the match.

For the second model, which predicts match win probability based on
current game score, the data frame used includes the current number of
games won by each player in addition to the player coefficient matrix
and the player1 win indicator. to get the

\hypertarget{data-exploration-and-visualization}{%
\subsection{Data exploration and
visualization}\label{data-exploration-and-visualization}}

This section will have descriptive statistics and visualizations of the
raw data. Use this section to reveal to the reader any interesting
relationships in the data, and convince the reader that the predictors
are related to the outcome. Visualizations are one of the most powerful
ways to communicate information to the reader, so it is important to
spend time producing clear, descriptive, eye-catching visualizations.

The package \texttt{pubtheme} has a \texttt{ggplot} theme called
\texttt{theme\_pub} that helps with making publication-quality
visualizations with \texttt{ggplot}. See
\url{https://github.com/bmacGTPM/pubtheme}. There are also several
templates there that you can copy, paste, and modify.

If you display a data visualization or some other summary of data,
discuss the significance of what you see. What does this tell you about
the data? What does it tell you that will help you with modeling? Do not
simply show a visualization for the sake of showing a visualization.

Since \texttt{echo=F} is the option chosen at the top, the default will
be to show the output but not the code:

\begin{verbatim}
[1] 2
\end{verbatim}

If you don't want to show the output either, you can use
\texttt{include=F}:

Nothing was shown above. If you want to force it to show the code for
some reason, you can override the default by putting the options
\texttt{echo=T} for this chunk.

\begin{Shaded}
\begin{Highlighting}[]
\DecValTok{1}\SpecialCharTok{+}\DecValTok{1}
\end{Highlighting}
\end{Shaded}

\begin{verbatim}
[1] 2
\end{verbatim}

However, since this is a formal report, you will likely not want to show
code.

\hypertarget{modelinganalysis}{%
\subsection{Modeling/Analysis}\label{modelinganalysis}}

Describe regression or classification model(s) used, or the analysis
that was performed. For each regression or classification model, discuss

\begin{itemize}
\tightlist
\item
  any assumptions that are made
\item
  the observation, the predictors, and the outcome (aka the rows of
  \(X\), the columns of \(X\), and \(y\))
\item
  what model you are using, and write out the model
\item
  what the coefficients mean (when applicable) and how this is related
  to your problem
\item
  appropriate measures of the performance of the model, such as measures
  of fit and predictive ability
\item
  whether or not you think the model is appropriate for this kind of
  data, and why, and
\item
  how easy/hard it is to interpret the results and explain them to
  either a technical or non-technical audience.
\end{itemize}

For other kinds of analysis, what you give is highly dependent on the
type of analysis. But in general, talk about assumptions, if they are
appropriate, how they might not be appropriate, and why you chose this
type of analysis.

\hypertarget{visualization-and-interpretation-of-the-results}{%
\subsection{Visualization and interpretation of the
results}\label{visualization-and-interpretation-of-the-results}}

Create visualizations of the results when appropriate, focusing on
visualizations that

\begin{itemize}
\tightlist
\item
  help describe aspects of the results that have real-world
  interpretation
\item
  help the reader understand how the model addresses the problem you are
  studying.
\end{itemize}

\textbf{Visualizations are one of the most powerful ways to communicate
information to the reader, so it is important to spend time producing
clear, descriptive, eye-catching visualizations.}

Discuss the results of the model or models you chose, and describe how
they are related to the problem statement or question that you were
trying to answer in the project.

If you have built multiple models or types of analysis, compare the
measures of performance and the ease of interpretability across models
or types of analysis, stating which model or models performed best, and
which model or models were most interpretable. Finally, decide which
model or type of analysis is best for your particular problem based on
some combination of performance and interpretability.

\hypertarget{conclusions-and-future-work}{%
\subsection{Conclusions and Future
Work}\label{conclusions-and-future-work}}

One or two paragraphs stating conclusions, recommendations, and ideas
for future work and improvements.

\end{document}
